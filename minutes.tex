\documentclass[12pt]{amsart}
\usepackage[letterpaper,hscale=0.8,vscale=0.8,twoside=false]{geometry}
\usepackage{hyperref}
%%%
\title{Social Media}
\author{Aaron Maxwell}
\date{\today}
\linespread{1.25}
%%%
\begin{document}
\maketitle
\tableofcontents
%%%
\section{Thursday, March 21, 2019}\label{mtg01}%Present: Conor, Aaron, Tehmina, Emily, Mehrdad, Dimitra; Absent: Kate, Gaurav
This meeting was called to introduce volunteers to the roles and requirements of the committee.
The following actions were outlined:
\begin{itemize}
\item Social media feed to emphasize Call for Panel Proposals, Emphasize Editorials, and Budget Symposium
\item Attach images, links, and other content to be stored in the drive
\item New volunteers are to read the on boarding document
\item Tehmina and Guarav to draft messages for enticing volunteers
\item Aaron to reorganize the drive and generate documents
\item Mehrdad and Dimitra to develop an organic document of message priorities for the committee
\end{itemize}
%%%
\section{Thursday, March 28, 2019}\label{mgt02}%Present: Emily, Tehmina, Louise, Kate, Gaurav, Mehrdad, Brenda, Dimitra; Absent: Aaron, Conor
The purpose of the meeting was to confirm the members were familiar with their roles.
Tehmina and Gaurav had drafted posts and sent them to Aaron.  Tehmina noted that it took about an hour for her to craft 3 tweets.
Other committee members have been reviewing CSPC?s social media feeds.
It was decided that, with the low number of volunteers on the committee right now that calls will be weekly.
The following actions were outlined:
\begin{itemize}
\item All drafted tweets will be stored in document for review and editing review and editing.
\item Images will be stored in a folder for use -- the updated poster for the 2019 Budget Symposium is available
\item Committee will start scheduling social media posts on Hootsuite, with focus on the call for volunteers, the 2019 Budget editorials, and the 2019 Budget Symposium.
\item Brenda will work on creating a calendar of events to share with the committee
\end{itemize}
%%%
\section{Thursday, April 4, 2019}\label{mgt03}%Present: Emily, Tehmina, Kate, Gaurav, Mehrdad, Brenda; Absent: Aaron, Louise, Conor
Tehmina may be able to take on the role of co-Chair in a few weeks.
Mehrdad noted that in the past, the social media strategy would be in place, but this year there are less volunteers on the committee.
Currently the amount of items for social media is relatively low but will pick up significantly once the CSPC 2019 Panels are finalized.
The social media posts for the past week were reviewed and the type of posts, frequency and content were discussed.
Of import is that our messaging, and especially our retweets, should be fact based, neutral, and non-partisan.
Retweeting of posts from our feed is also needed.
A \href{https://drive.google.com/open?id=1rrq5VrNb8jYogH0FOZP76n3koRmJKfNqVeFVJnFQLDM}{document} used for drafting and recording social media posts was shared on the drive.
There were several action items discussed:
\begin{itemize}
\item Where possible the messages should include an image: either poster, infographic, or conference pictures.
\item For the next two weeks until April 15:
\begin{itemize}
	\item daily messages for call for panel proposals 
	\item daily messages for budget symposium
	\item 2-3 messages for editorials
\end{itemize}
\item We should also be sending messages for calling for new volunteers to join the Social Media \& Outreach Committee.
\item We should also be retweeting interesting developments in science and innovation policy
\item There will be a post on all social media regarding the Environment Canada report released on Monday.
\item Brenda will put the banner for each in \href{https://drive.google.com/open?id=14IQquZwG4s8bjI-TdxFss1Dqcx8rVPkr}{this folder}.
\item Pictures from the 2018 conference can be downloaded from the \href{https://sciencepolicy.ca/cspc-2018-photos}{website}.
\item The Committee will divide the social media postings schedule with 1 volunteer for each day.
\end{itemize}
%%%
\section{Thursday, April 11, 2019}\label{mgt04}%Present: Louise, Aaron, Gaurav, Mehrdad, Dalini; Absent: Tehmina, Conor, Emily, Kate
The primary focus of this call was to determine why the committee has been so lacklustre in its operation.
The biggest problem, it seems, is a lack of coordination to the volunteers.
Aaron made it clear that going forward there will be more communication between himself and the volunteers so that they are clear on their tasks.
One clear goal for the committee is to hit the 3--5 social media messages per day, rather than the 1 per day currently being pushed.
There are now a few images on the Drive that can be used to append to social media messages.
Aaron has cropped and re-sized the photos of volunteers with Dr.~Mona Nemer and the Hon.~Kirsty Duncan.
The following action items were identified:
\begin{itemize}
\item Aaron will schedule a daily panel promotion countdown on Twitter to emphasize the deadline on Monday: 
\item Emily to schedule two messages next week on the budget editorials, and each message should append the author collage in the drive.
\item Kate to schedule two messages next week with regards to the Re\$earch Money conference happening next week.
\item Our messages asking for volunteers should emphasize joining \emph{our} committee.
\item Louise to craft social media messages on science policy youth awards of excellence --- Aaron to send info.
\item Gaurav to edit his social media messages to emphasize Social Media Committee -- Aaron to send notes.
\item Lily to read the on boarding document on the drive.
\item Tehmina and Fatima to help with monitoring the social media feed next week.
\end{itemize}

\end{document}