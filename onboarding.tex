\documentclass[11pt]{amsart}
\usepackage[letterpaper,hscale=0.8,vscale=0.8,twoside=false]{geometry}
\usepackage{hyperref}
\newcommand{\currentChair}{Aaron Maxwell}
\newcommand{\currentAdmin}{Dimitra Chronopoulos}
%%%
\title{Outreach \& Social Media Committee: Onboarding}
\author{Aaron Maxwell -- Current Chair \\ Conor Meade, Greg Hodgson, \& Rebecca Solomon -- past Co-chairs}
\date{\today}
\linespread{1.25}
%%%
\begin{document}
\maketitle
\section{Introduction}
Thank you for expressing interest in volunteering for the Outreach and Social Media Committee for the Canadian Science Policy Conference.
In the following pages, you will gain some insight into how the Committee operates, what its broad goals are, and how you can contribute to achieving those goals.
It may seem like a lot to take in, but the best way to learn quickly is to jump right in: sign up for tasks, offer to help whenever you can, and never be shy about asking the more experienced volunteers for guidance.
%%%
\subsection{History}
The Canadian Science Policy Conference (CSPC) was founded in 2009 by a diverse group of young and passionate professionals from industry, academia, and science-based governmental departments.
CSPC serves as an inclusive, non-partisan and national forum uniting stakeholders, strengthening dialogue, and enabling action with respect to current and emerging issues in national science, technology, and innovation policy.
CPSC is a grassroots initiative that thrives on a dynamic and responsive conference structure focused on best serving the needs of the science policy community.
Year after year, it has flourished thanks to the efforts of more than 100 volunteers dedicated to engaging members from research, technology, policy, and the general public in thoughtful discussion to improve governance of issues important to Canadian society.
The event has experienced tremendous expansion over the course of its 7 years, growing to more than 1,400 attendees.
This would not have been possible without the support of over 200 companies, institutions, and agencies, as well as the Advisory and Honorary Committees comprising of more than 70 field-leading experts.
%%%
\subsection{Committee Mission}
The mission of the Outreach and Social Media Committee (OSMC) is to drive awareness and interest in the Canadian Science Policy Conference and in all activities related to the Canadian Science Policy Centre.
The broad mission statement of CSPC is the following:
\begin{itemize}
\item Act as a hub promoting ongoing national dialogue in support of more effective coordination and collaboration across all stakeholders, nurturing a broad and evolving community of policy thinkers
\item Build capacity by introducing and fostering initiatives to train the next wave of leaders in science, technology, and innovation policy
\item Advance science policy research by enabling a framework for collaborative, multi-sectoral assessment of science and innovation policies in Canada
\end{itemize}
Thus, our main goal is to actively promote the conference, and in doing so, continue to enable CSPC to achieve its mission.
We will also be called upon to promote many other activities of both the Centre and the Conference, such as online editorials and other publications, smaller CSPC events, and serving as an online information hub for the Canadian and Global science policy communities.
%%%
\section{Operation}
The following will outline how the committee operates in terms of meetings, task assignment, and documentation.
%%%
\subsection{Meetings}
Our goal is to have the Committee meet bi-weekly, on \emph{Thursdays} at \textbf{12:00 PM EST}, using the \href{https://www.zoom.us/}{Zoom} tele-conferencing software.
The connection information is:
\begin{itemize}
\item Social Media Meeting Room ULR: https://zoom.us/j/126351747
\item Social Media Meeting Room App: 126-351-747
\item General Meeting Room URL: https://zoom.us/j/4107793734
\item General Meeting Room App: 410-779-3743
\end{itemize}
Weekly attendance is not mandatory, but it is important that you notify the current chair, \currentChair, or the current CSPC administrator, \currentAdmin.
At each meeting we will determine our next set of tasks and assess our success (or lack thereof) on our previously set tasks.
%%%
\subsection{Documentation}
Our most important planning and tracking document is the \href{https://docs.google.com/spreadsheets/d/1lpFBDv1iNDUWX_q2PlV3-wX9bY-DUmBBHCQLArtSenY/edit?usp=sharing}{Roles and Responsibilities} spreadsheet in \href{https://drive.google.com/drive/folders/1JjcAsWjjbja4auyTV0g8y4WW9DR7mqMj}{Google Drive folder}.
This is where we all sign up for tasks and projects, track their completion, and generally keep the committee running smoothly.
The document has several sheets, each corresponding to a different area of responsibility:
\begin{itemize}
\item \textbf{Content Strategy}: Outlines the content focus for each week, the number of posts to our social media channels, and account ownership
\item \textbf{Panel Promotion}: Outlines our targeted promotion of conference panels when they are announced
\item \textbf{Community Friends}: Outlines are outreach and promotional efforts with CSPC community friends (promotional partners)
\item \textbf{Partners}: Same as above, but for financial supporters of CSPC.
\end{itemize}
There will other documents in the Google Drive folder, such as our analytic tracking of social media impact, specific messaging for panels, and so on.
%%%
\subsection{Login Information}
Our channels are the following:
\begin{itemize}
\item Twitter: @sciencepolicy
\item Facebook: @canadiansciencepolicy
\item LinkedIn (`Canadian Science Policy Centre'): https://ca.linkedin.com/company/canadian-science-policy-centre
\end{itemize}
Although it is exceptionally bad practice to store readable passwords, we do so because access to the account must be shared.
\begin{table}
\centering
\begin{tabular}{|c|c|c|}\hline
Account & User & Password \\\hline
Hootsuite and Twitter & socialmedia@sciencepolicy.ca & 5FOkpIgfzEA \\\hline
E-mail & outreach@sciencepolicy.ca & 20outreach14 \\\hline
E-mail & socialmedia@sciencepolicy.ca	& CSPC2015 \\\hline
gccollab.ca & outreach@sciencepolicy.ca & 20outreach18 \\\hline
\end{tabular}
\end{table}
%%%
\section{Social Media Content Guidelines}
The best way to learn how to create good content for our Twitter, Facebook and LinkedIn is to review previous postings and align your content with what has already been posted.
This will help you get a feel for the sort of content, the tone, and the \emph{voice} of the CSPC brand.
Especially for new volunteers, a senior seasoned volunteer will help you create content and review your work to provide you with as much learning experience as you require.
The mechanics of posting and scheduling content are pretty straightforward, and HootSuite offers plenty of training resources, such as \href{https://blog.hootsuite.com/how-to-schedule-tweets/}{How To Schedule Tweets}
One of the previous co-chairs, Rebecca Solomon, has also created a PDF that provides a visual walkthrough that is located in the Google Drive folder.\\
\indent When Twitter and other similar social media platforms were introduced, a popular way of archiving and sorting particular topics was through the use of a \emph{hashtag}: an octothorp (\#) followed by a keyword or key-phrase.
Our main hashtag is that of the current years conference, \#CSPC2019, but other appropriate hashtags are:
\begin{itemize}
\item SciPolicy
\item SciComm
\item CdnSci
\item CdnInnovation
\item CdnTech
\item ScienceDiplomacy
\item CdnPoli
\item CdnPSE
\end{itemize}
Keep in mind that hashtags should only be used when appropriate - for example, if the message does not concern causes advancing Women in Science, Technology, Engineering and Math (STEM), then the \#WomenInSTEM hashtag should not be included.
Content readers\footnote{It is also suggested that, for twitter, handles be given at the end of the message.} for persons with visual disabilities have an easier time with hashtags that are written in camel case, where the next letter of each word is capitalized.
Furthermore, LinkedIn does not use hashtags in the same manner as the other social media platforms, so please refrain from using them there.
Our goal with our messages is to `add value', in other words, to give our readers a reason to keep reading our content.
Our communication is 1\% `selling' and 99\% informing, connecting, and entertaining.
As mentioned previously, if you are unsure you can always ask for help.
%%%
\subsection{Long Term Goals}
We want to always ensure that our social media messaging is effective by taking a targeted and thoughtful approach to our posts.
In other words, we want to prevent social media posts that are too frequent, unfocused, and non-strategic.
Our primary topics each year are (in no particular order): Partner promotion, Community Friends, Conference Registration, and Conference Participation.
Secondary topics include science policy news, our CSPC editorials, and any other CSPC events.
However, other topics of interest may also crop up, and the key goals we should keep in mind are as follows:
\subsubsection{What is our ideal messaging frequency and distribution?}
In other words, how often should we post to our social media channels, and what percentage of messages should be allocated to each topic?
In terms of Facebook and LinkedIn, our historical approach has been to limit messages on each channel to 3--5 \emph{per week}.
For Twitter, our ideal approach is to limit messages to 3--5 \emph{per day}.
An article in \href{http://follows.com/blog/2016/04/times-day-post-twitter}{2016} noted that with the introduction of Twitter's feed filter, it can be difficult to get all of our content into our followers' feed:
\begin{verbatim}
Recently, Twitter introduced the filtered feed.  It's not quite so filtered as
Facebook's feed, but it's still an engine that recommends content to you rather than
just presenting you with the tweets that have been made since your last visit.
Analysis suggests that you should limit yourself to three tweets per day.  If this
seems low to you, well, it kind of is.  Most people think five is a more comfortable
number, and it is.  Again, there's nothing really wrong with posting more than three
times per day; you just have to be aware that the 4th and later tweets will tend to
have lower engagement compared to the first three, and that the more you post, and
the more often you do it, the less you're going to get in returns.
\end{verbatim}
Historically, Aaron Maxwell has been tracking our social media engagement via the Twitter API, and his results verifies this.
An industry sectoral \href{https://blog.hubspot.com/marketing/social-media-frequency-industry-benchmarks}{benchmarking} study finds that the majority of social media posts should be limited to less than 3 per day.
More importantly this is the volume of posts that other similar organizations to CSPC use, such as CIFAR, Research Money, the Conference Board of Canada, and the AAAS.\\
\indent As for topic allocation per week, our main goal is to ensure that \emph{at least} 55\% of our messages should be targeted towards CSPC promotion: panel proposals, speaker highlights, and calls for registration.
Promotion of Community Friends and Partners should encompass 30\% of our messaging, with the other 15\% allocated to the rest of the topics.
\subsubsection{Implementation}
For the most part, implementing these goals should be quite straight forward.
HootSuite allows us to pre-schedule content, and so that should be leveraged as much as possible.
Especially for CSPC 2019 panel announcements and promotion, it should be easy enough to schedule messages weeks in advance.
One of the sheets in the Roles and Responsibilities document will allow us to keep track of messaging frequency and distribution.
The primary responsibility for the weekly owners of our social media channels will thus be tasked with delivering content that varies with the day to day: promotings cience policy news and our editorial content, engaging with our followers, and retweeting partners and friends.
Weekly managers will also review the scheduled content to ensure it is evenly spaced.
%%%
\subsubsection{Posting from Personal Accounts}
We highly encourage you to post or re-post information from your social media account if you so choose.
However, be cautious to never speak on behalf of CSPC from your personal accounts.
%%%
\subsection{Community Friends and Partners}
One of our tasks is to coordinate with our partners and sponsors to cross-promote content.
There is a Community Friends strategy guide in the Google Drive folder that goes into more detail about how this is achieved.
\end{document}
%Conor Meade conor.meade@gmail.com
%Rebecca Solomon solomon.rb@gmail.com
%Aaron Maxwell	ajmax@paladin.ai
%Fatima Tokhmafshan	fatima.tokhmafshan@gmail.com
%Emily Jacobs	eejcbs@gmail.com
%Kate Sedivy-Haley	kate@hancocklab.com
%Guarav Dhanda gdhanda02@gmail.com
% Louise Moyle louise.moyle@utoronto.ca
